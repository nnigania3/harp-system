\setcounter{equation}{0}

\chapter{Design Overview and Tool Chain}
\label{chap: Design Overview and Tool Chain}

%%%%%%%%%%%%%%%%%%%%%%%%%%%%%%%%%%%%%%%%%%%%%%%%%%
\section{ System Overview }
The aim of this work was to design a custom GP-GPU. Since we wanted the whole system to be customized, a custom ISA (\cite{schreier}) called `HARP' was used for this work. The ISA is a MIPS based ISA with support for several customizations which will be discussed in the next section. The key features are predication support, configurable instruction width, no. of general purpose and predicate registers and vector support. 
\\
As for the design, the main components are the core, the cache and the memory controller- as can be seen in the Figure~\ref{fig:system_overview}. The Figure~\ref{fig:system_overview} is a very simple design which we have used just to demonstrate different components. Each of these components will be discussed separately in the next section. We will also look at the different possible systems we can design in chapter 4. The core which implements the main pipeline and the memory management unit was written in `CHDL' \cite{schreier}. The cache was written in verilog but a few different versions also exist. The memory controller IP was generated using Altera's FPGA tools (Quartus II, \cite{schreier}). Even though we have built a system using a particular configuration, we can use this flow to try many different scenarios. For example if we want to built a system which uses a new kind of memory system (example HMC \cite{schreier} ) all we would have to do is to replace the DDR2 controller which we are using right now with the IP of the new memory controller and everything else will remain the same. We can also add new instructions to the ISA and add support to the core (like support for multiple warps like in GPUs) keeping everything in the uncore part unchanged. 
\\
% LEFT, BOTTOM, RIGHT,TOP
 \figput{system_overview}{6.0in}{0.0in 0.5in 0.0in 0.2in}{Basic system components}

As we can see in Figure~\ref{fig:system_overview}, our design supports multiple cores and each core can support SIMD hence creating a GPGPU. To enable support for multiple cores we have also designed the arbitrator which sends requests from the private caches of each core to the shared cache. Also similar to the way GPGPUs are designed; our design does not support coherence among the private caches of each of the cores. So the programmer should be aware of this fact and must write code accordingly. For the current work most benchmarks were written in HARP assembly but one of the future tasks as part of this project is to design a CUDA / OpenCL to HARP translator which would the allow us to run more general commercially available applications.
%%%%%%%%%%%%%%%%%%%%%%%%%%%%%%%%%%%%%%%%%%%%%%%%%%
\section{ CHDL }
CHDL \cite{schreier}. CHDL is the custom environment used to write HDL (hardware description language) in C++. It has similarities to system C. We used CHDL so that as previously mentioned the same code can be used for the FPGA flow as well as the simulation flow. CHDL is implemented as a source to source translator which translates high level C++ to netlist/Verilog HDL. CHDL is basically a set of C++ libraries used to support correct translation of code to its Verilog equivalent. Even though we write in C++ we still have to think to some extent as to how the verilog code will be generated and keep in mind that assignments to wires (represented a vector of Boolean) are continuous and we can assign the wire only once.
More details of CHDL along with examples can be found in Appendix A.
There are several advantages of taking the CHDL approach, as we write code in C++ it becomes easy to describe complex functions using simple code as compared to Verilog. Also since we already have a library of commonly used functions we can write code quickly, for example we have implementations of multiplexers, decoder, encoders, state machines etc. Obviously the down side of using this approach according to me is that since we can describe lot of complex functionality using this approach hence it might not always generate the most optimal verilog code. Although I expect the Verilog compiler to optimize the code and remove redundancies but I think we would be able to generate better code if we write directly in Verilog. The another biggest downside of using this approach is that it is very hard to debug, as the signal /variable names are lost in the code generated and we have to specify signals using special debug statements if we want to preserve the name to help us in debugging. %The verilog code is also huge in size and unreadable.
\\
Figure~\ref{fig:chdl_simple_example} shows a simple example of code written using CHDL and compared with its verilog equivalent. The verilog code generated from CHDL is about 106 lines compared to about 10 lines in verilog. As can be seen this is a code for simple 5 bit counter. The verilog code is implied. As for the CHDL version, we represent the counter with as a vector of bool with 5 bits. The `Reg' statements translates to \begin{quotation}    
\centering always@(posedge clk) output \textless= input; 
\end{quotation} 
Similarly if we want to implement basic operations we can directly use statements like that shown in Figure~\ref{fig:chdl_alu_example}. This figure shows the implementation of a basic ALU where the final output is that coming out from the multiplexer which selects the correct output based on the  `op\_select'  which acts as the selector.

% LEFT, BOTTOM, RIGHT,TOP
 \figput{chdl_simple_example}{5.0in}{0.0in 1.5in 0.0in 1.0in}{Simple Counter in CHDL and Verilog}

% LEFT, BOTTOM, RIGHT,TOP 
\figput{chdl_alu_example}{5.0in}{0.0in 0.5in 0.0in 0.5in}{Basic ALU in CHDL}

%%%%%%%%%%%%%%%%%%%%%%%%%%%%%%%%%%%%%%%%%%%%%%%%%%
\section{FPGA board and Development Tools}
The FPGA board used for the studies and experiment is a Terasic DE3 board. It uses an Altera FPGA for prototyping. The FPGA used is a Stratix III 3SL150 FPGA with the following properties 142,000 logic elements (LEs); 5,499K total memory Kbits; 384 18x18-bit multipliers blocks; 736 user I/Os. The DE3 board also has support for DDR2 RAM, USB, leds, 7-segement display and connectors to stack multiple boards. Figure~\ref{fig:fpga_board} shows a picture of the FPGA board used for this work.

We used Altera's FPGA tools for development purposes. The Quartus 11.3 tool was used to synthesize the HDL and program the FPGA device via USB. We used ModelSim for the RTL and gate level simulation experiments. As for other tools used we also used open source tools like `iVerilog' to compile our verilog code and dump wave signals to a file which we then viewed using `gtkwave'.

The DDR2 was used along with Altera's DDR2 memory controller IP. More about the DDR2 controller will be discussed in the chapter 3. The Quartus tool with a built in IP generator called `Megacore wizard' was used to generate the memory controller IPs along with other commonly used IPs like PLLs for managing clocks.

% LEFT, BOTTOM, RIGHT,TOP
 \figput{fpga_board}{4.0in}{0.0in 0.0in 0.0in 0.0in}{Altera Stratix III Board}
%%%%%%%%%%%%%%%%%%%%%%%%%%%%%%%%%%%%%%%%%%%%%%%%%%
