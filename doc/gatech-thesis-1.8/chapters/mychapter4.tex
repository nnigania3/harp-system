\setcounter{equation}{0}
 
\chapter{Measurement Results and Analysis}
\label{chap: Simulation Results }
%%%%%%%%%%%%%%%%%%%%%%%%%%%%%%%%%%%%%%%%%%%%%%%%%%
\section{Board and Test Environment}

%\TODO{this part is a repeat}.
The FPGA board used for the studies and experiment is a Terasic DE3 board. It uses an Altera FPGA for prototyping. The FPGA used is a Stratix III 3SL150 FPGA with the following properties: 142,000 logic elements (LEs), 5,499K total memory Kbits, 384 18x18-bit multipliers blocks and 736 user I/Os. The DE3 board also has support for DDR2 RAM, USB, leds, 7-segement display and connectors to stack multiple boards. Figure~\ref{fig:fpga_board} shows a picture of the FPGA board used for this work. We use a 1GB DDR2 DIMM. The benchmarks as mentioned earlier in Table~\ref{table:apps} were all written in Harp assembly.

To measure the performance we calculated IPC(Instructions Per Cycle). The number of instructions was calculated using the Harptool emulator. Number of cycles can be easily measured on the simulator but to measure the actualy number of cycles on the FPGA board we created hardware counters in our design. We start counting when all components are initialized and stop when we reach the end of our benchmark. To indicate correctness we display three signals on the board to indicate test complete, pass and fail for the core. The checksum logic for the pass signal is defined for each benchmark in our test module, say for example we check the final sum of `array sum' application with the known answer. We also have two more signals to show DDR2 controller initilization pass/fail. The number of cycles information from the hardware counter is displayed on the 7-segment display on the board.
%%%%%%%%%%%%%%%%%%%%%%%%%%%%%%%%%%%%%%%%%%%%%%%%%%
\section{Results}

The basic parameters for the core and cache are set as shown in Table~\ref{table:core_config}
\begin{table}[!htbp]
  \centering
  \begin{tabular}{|l|c|}
    \hline
    \multicolumn{2}{|c|}{Core Configuration} \\
    \hline
FPGA Device:				& StratixIII (EP3SL150F1152C2)\\
FPGA flow tool:				& Quartus II 13.0.1\\
Core Operating Freq:			& 62.5 MHz\\
DDR2 Operating Freq:			& 125 MHz\\
Instruction Width:			& 32 bits\\
L1 / L2 cache:				& 16KB / 128KB\\
General Registers:			& 16 32-bit Regs\\
Predicate Registers:			& 16\\
SIMD Lanes:				& 8\\
Instruction ROM:			& 1024KB\\
%Vector Registers: 			&  (16 for SIMD)\\
    \hline
  \end{tabular}
  \caption{Core Configuration.}
  \label{table:core_config}
\end{table}

%%%%%%%%%%%%%%%%%%%%%%%%%%%%%%%%%%%%%%%%%%%%%%%%%%
After creating the basic building blocks of our system, we tried to run performance and functional tests on a few example designs. Below we discuss three versions of the Harp Core and its performance for simple micro-benchmarks.\\

\noindent\textbf{Single 1-Lane Harp Core:}\\
First we ran the benchmarks on a simple 1-lane Harp Core. For the benchmarks ran, we did not see much benefit of using a complex load store queue as the applications had a dependent instruction following a load instruction which would stall the pipeline until the dependent load is serviced. Hence to make the design simple we removed the complex load/store queue so the core now sends blocking requests to the memory subsystem, though independent instructions can still keep flowing in the pipeline. Table~\ref{table:perf1} shows the performance and Table~\ref{table:fpga_util1} shows the logic utilization of this system. The low IPC is because almost all applications have a chain of dependent instructions.  

\begin{table}[!htbp]
  \centering
  \begin{tabular}{|l|c|c|c|c|}
    \hline
Benchmark		&Instructions 	&Cycles		&IPC		&Description\\
    \hline
Array Sum:		&2912		&9598		&0.3034 	&Sum 240 numbers \\
Sieve of Eratosthenes:	&1611		&5504		&0.2926 	&Prime numbers in 1 to 100\\
Bubble sort:		&799		&2593		&0.3081 	&Sort 0-9\\
Matrix multiplication:	&6082		&19855		&0.3063 	&Multiply 8x8 matrices\\
    \hline
  \end{tabular}
  \caption{Single Core 1-Lane Performance}
  \label{table:perf1}
\end{table}

\begin{table}[!htbp]
  \centering
  \begin{tabular}{|l|c|}
    \hline
    \multicolumn{2}{|c|}{Logic Utilization} \\
    \hline
FPGA Logic Utilization:		& 22\% 	\\
   Combinational ALUTs:		& 13\% 	(14,850 / 113,600 ALUTs)\\
   Dedicated Logic Registers:	& 12\% 	(14,850 / 113,600 ALUTs)\\
   Memory ALUTs:		& 3\% 	(1,499 / 56,800 ALUTs)\\
Total block memory bits:	& 27\%  (1,505,792 / 5,630,976 )\\
Total pins:			& 25\% 	(189 / 744)\\
    \hline
  \end{tabular}
  \caption{Logic Utilization Of A Single 1-Lane Core}
  \label{table:fpga_util1}
\end{table}

%%%%%%%%%%%%%%%%%%%%%%%%%%%%%%%%%%%%%%%%%%%%%%%%%%
\noindent\textbf{Dual 1-Lane Harp Core}\\
Next we instantiate two of the 1-lane Harp cores above to form a multi-core system. Table~\ref{table:perf2} shows the performance and Table~\ref{table:fpga_util2} shows the logic utilization of this system.

\begin{table}[!htbp]
  \centering
  \begin{tabular}{|l|l|c|c|c|}
    \hline
Core-1 Benchmark	&Core-2 Benchmark	&Instructions 	&Cycles		&IPC	\\
    \hline
Array Sum   	&Sieve of Eratosthenes 			&4523		&9598		&0.4712	\\
Bubble Sort   	&Sieve of Eratosthenes    			&2410		&5504		&0.4378	\\
Array Sum		&Matrix multiplication		&8994		&19857		&0.4529	\\
Matrix multiplication&Matrix multiplication	&12164		&19857		&0.6125 \\
    \hline
  \end{tabular}
  \caption{Dual Core 1-Lane Performance}
  \label{table:perf2}
\end{table}

\begin{table}[!htbp]
  \centering
  \begin{tabular}{|l|c|}
    \hline
    \multicolumn{2}{|c|}{Logic Utilization} \\
    \hline
FPGA Logic Utilization:		& 30\% 	\\
   Combinational ALUTs:		& 19\% 	(21,945 / 113,600 ALUTs)\\
   Dedicated Logic Registers:	& 17\% 	(19,027 / 113,600 ALUTs)\\
   Memory ALUTs:		& 3\% 	(1,499 / 56,800 ALUTs)\\
Total block memory bits:	& 29\%  (1,647,104 / 5,630,976 )\\
Total pins:			& 25\% 	(189 / 744)\\
    \hline
  \end{tabular}
  \caption{Logic Utilization Of A Dual 1-Lane Core}
  \label{table:fpga_util2}
\end{table}

We can clearly see in Table~\ref{table:perf2} the benefits of using multiple cores on performance. Also we can see how by using a simple design of the core and having slightly more complex cache design helps us to keep the logic utilization to the minimum (scalable design). Given the above logic utilization we can easily instantiate more than 8 cores on the FPGA device.
Since there is no sharing and the applications are not memory internsive the performance scales linearly with number of cores for the above applications when we replicate the application across multiple cores as we can see in the results for matrix multiplication. \\

%%%%%%%%%%%%%%%%%%%%%%%%%%%%%%%%%%%%%%%%%%%%%%%%%%
\noindent\textbf{Single SIMD 8-Lane Harp Core}\\
We also design and run an 8 Lane SIMD Core on the FPGA board. We used simple applications initially to test the complex coalescing unit. Once that was done we tried to run a compute internsive matrix multiplication code on the board. Comparing the performance numbers we can clearly see the advantage of using SIMD. The effective IPC would be about 8x of this reported value as the same instruction is executed across all the eight lanes. This comes at the cost of much higher logic utilization due to the coalescing unit and the duplication of ALU blocks and register files for the SIMD core. Table~\ref{table:perf3} shows the performance and Table~\ref{table:fpga_util3} shows the logic utilization of this system.

%Benchmarks Results:
\begin{table}[!htbp]
  \centering
  \begin{tabular}{|l|c|c|c|c|}
    \hline
Benchmark		&Instructions 	&Cycles		&IPC		&Description\\
    \hline
Matrix Multiplication:	&929		&2881		&0.3224 	&Multiply matrices of size 8x8\\
Coalesced Vector Sum:	&399		&1068		&0.3735 	&Sum 240 numbers\\
Un-Coalesced Vector Sum:&399		&1300		&0.3069 	&Sum 240 numbers\\
    \hline
  \end{tabular}
  \caption{8-Lane SIMD Core Performance}
  \label{table:perf3}
\end{table}

%Logic Consumption:
\begin{table}[!htbp]
  \centering
  \begin{tabular}{|l|c|}
    \hline
    \multicolumn{2}{|c|}{Logic Utilization} \\
    \hline
FPGA Logic Utilization:		& 54\% 	\\
   Combinational ALUTs:		& 37\% 	(42,321 / 113,600 ALUTs)\\
   Dedicated Logic Registers:	& 24\% 	(27,517 / 113,600 ALUTs)\\
   Memory ALUTs:		& 3\% 	(1,499 / 56,800 ALUTs)\\
Total block memory bits:	& 27\%  (1,505,792 / 5,630,976)\\
Total pins:			& 25\% 	(189 / 744)\\
    \hline
  \end{tabular}
  \caption{Logic Utilization Of A Single 8-Lane SIMD Core}
  \label{table:fpga_util3}
\end{table}

Using SIMD does give us performance benefits as can be seen in  Table~\ref{table:perf3}. We also dont see a big difference in the performance of coalesced and uncoalesced array sum applications due to the fact that we have a non-blocking cache and a load store queue. For the above tests a maximum of 4 requests are being served simultaneously but can be easily increased by changing the parameter for MSHR entries and the FIFO size for cache-DDR interface.

The aim of all the above experiments was not only to show obvious benefits due to SIMD or multiple cores. We can always choose to build a system and select an  application to show really good performance benefit. But rather the aim of this chapter was to estabilish credibility and show the felxibility offered by the whole design. Depending on the resource constraints and performance goals we can create a heterogeneous system where we have the choice to select the type of core (simple/complex/SIMD) and the number of cores.
%%%%%%%%%%%%%%%%%%%%%%%%%%%%%%%%%%%%%%%%%%%%%%%%%% 

\section{Prototyping a big.LITTLE Heterogeneous Harp System}
The above section demonstrates some of the conventional systems that can be designed using our tool set. In order to highlight the flexibility offered in quickly prototyping new systems we tested a heterogeneous system consisting of a big SIMD core along with many small cores. This system has similarities to the big.LITTLE systems proposed by ARM\cite{biglittle}. This system also fits the description of a possible future system consisting of an HMC. An HMC has layers of memory stacked on top of a logic layer which can be an ideal place to implement tiny processors for computing close to the memory. The vault in an HMC is one vertical division with its own logic and memory stack. Hence a vault can be one such tiny core. In our system the smaller cores simulate a vault. They all have their own memory close to them alleviating the need for a cache. All the small cores work independently with a big SIMD core which is running a compute intensive application like matrix multiplication. Since there can be many such cores we look at the examples of a 1 big core along with two, six and eight little cores. Table~\ref{table:perf4} shows the performance of various heterogeneous systems tested and Table~\ref{table:fpga_util4} shows the logic utilization of these systems.

%Benchmarks Results:
\begin{table}[!htbp]
  \centering
  \begin{tabular}{|l|c|c|c|l|}
    \hline
Core Configuration		&Total Instructions 	&Cycles		&IPC		&Description\\
    \hline
1 Big - 2 Little:		&2527 	&2881		&0.8771 	&Big core: matmul\\
								&			&			&			&Little cores: bubble sort 0-9\\
\hline
1 Big - 6 Little:		&5723 	&3102		&1.8444 	&Big core: matmul\\
								&			&			&			&Little cores: bubble sort 0-9\\
\hline
1 Big - 8 Little:		&7321		&3041		&2.4074 	&Big core: matmul\\
								&			&			&			&Little cores: bubble sort 0-9\\
    \hline
  \end{tabular}
  \caption{Heterogeneous Harp System Performance}
  \label{table:perf4}
\end{table}

%Logic Consumption:
\begin{table}[!htbp]
  \centering
  \begin{tabular}{|l|c|c|c|}
    \hline
%    \multicolumn{4}{|c|}{Logic Utilization} \\
%    \hline
FPGA Resource					& 1 Big - 2 Little	&1 Big - 6 Little		&1 Big - 8 Little\\
    \hline
FPGA Logic Utilization:		& 62\% 				& 80\% 				& 88\% \\
   Combinational ALUTs:		& 44\% 				& 58\%				& 65\% (73,590/113,600)\\
   Dedicated Logic Registers:	& 27\%			& 31\%				&33\% (37,649/113,600)\\
   Memory ALUTs:				& 3\%					& 1\%					& 1\% (302/56,800)\\
Total block memory bits:	& 27\%				& 28\%				& 28\% (1,590,474/5,630,976)\\
    \hline
  \end{tabular}
  \caption{Logic Utilization Of big.LITTLE Harp Systems}
  \label{table:fpga_util4}
\end{table}

The above results showcase the benefits of having these kinds of heterogeneous systems as one of the many possible options. We can also easily see how scalable and flexible our tool chain is in designing these systems for future research. 
%%%%%%%%%%%%%%%%%%%%%%%%%%%%%%%%%%%%%%%%%%%%%%%%%%
