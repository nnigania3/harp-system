\setcounter{equation}{0}
 
\chapter{Experiment Results }
\label{chap: Simulation Results }
%%%%%%%%%%%%%%%%%%%%%%%%%%%%%%%%%%%%%%%%%%%%%%%%%%
\section{Board and Test Environment}

%\TODO{this part is a repeat}.
The FPGA board used for the studies and experiment is a Terasic DE3 board. It uses an Altera FPGA for prototyping. The FPGA used is a Stratix III 3SL150 FPGA with the following properties 142,000 logic elements (LEs); 5,499K total memory Kbits; 384 18x18-bit multipliers blocks; 736 user I/Os. The DE3 board also has support for DDR2 RAM, USB, leds, 7-segement display and connectors to stack multiple boards. Figure~\ref{fig:fpga_board} shows a picture of the FPGA board used for this work. We use a 1GB DDR2 DIMM. The benchmarks as mentioned earlier were all written in Harp assembly and results of their run on the board are shown below.

%%%%%%%%%%%%%%%%%%%%%%%%%%%%%%%%%%%%%%%%%%%%%%%%%%
\section{Benchmarks Ran and Results}

The basic parameters for the core and cache are set as shown in Table~\ref{table:core_config}
\begin{table}[!htbp]
  \centering
  \begin{tabular}{|l|c|}
    \hline
    \multicolumn{2}{|c|}{Core Configuration} \\
    \hline
Core Operating Freq:			&62.5 MHz\\
DDR2 Operating Freq:			&125 MHz\\
Instruction Width:			&32 bits\\
L1 / L2 cache:				& 16KB / 128KB\\
General Registers:			& 16 32-bit Regs\\
Predicate Registers:			& 16\\
SIMD Lanes:				& 8\\
Instruction ROM:			& 1024KB\\
%Vector Registers: 			&  (16 for SIMD)\\
    \hline
  \end{tabular}
  \caption{Core Configuration.}
  \label{table:core_config}
\end{table}

%%%%%%%%%%%%%%%%%%%%%%%%%%%%%%%%%%%%%%%%%%%%%%%%%%
After creating the basic building blocks of our system and we tried to run performance and functional tests on a few different basic variation of the design. Below we discuss 3 versions of the Harp Core and its performance for simple micro-benchmarks.

\textbf{Single 1-Lane Harp Core:}\\
First we ran the benchmarks on a simple 1-lane Harp Core. For the benchmarks ran, we did not see much benefit of using a complex load store queue as the applications had a dependent instruction following a load instruction which would stall the pipeline until the dependent load is serviced. Hence to make the design simple we removed the complex load/store queue so the core now sends blocking requests to the memory subsystem, though independent instructions still keep flowing in the pipeline. Table~\ref{table:perf1} shows the performance of and Table~\ref{table:fpga_util1} shows the logic utilization of this system.
\begin{table}[!htbp]
  \centering
  \begin{tabular}{|l|c|c|c|c|}
    \hline
Benchmark		&Instructions 	&Cycles		&IPC		&Description\\
    \hline
Vector Sum:		&2912		&9598		&0.3034 	&Multiply 2 matrices of size 16x16 \\
Sieve of Eratosthenes:	&1611		&5504		&0.2926 	&finding prime numbers between 1 to 100\\
Bubble sort:		&799		&2593		&0.3081 	&Sort Numbers 1-10 using bubble sort\\
Matrix multiplication:	&6082		&19855		&0.3063 	&Multiply two 8x8 matrices\\
    \hline
  \end{tabular}
  \caption{Single Core 1-Lane Performance}
  \label{table:perf1}
\end{table}

Logic Consumption:
\begin{table}[!htbp]
  \centering
  \begin{tabular}{|l|c|}
    \hline
    \multicolumn{2}{|c|}{Logic Utilization} \\
    \hline
FPGA Utilization:				&4\%  (3,661 / 113,600 ALUTs)\\
Combinational ALUTs:			&13\%       (14,446 / 113,600 ALUTs)\\
Total block memory bits:	&27\%   (1,505,792 / 5,630,976 )\\
    \hline
  \end{tabular}
  \caption{Logic Utilization.}
  \label{table:fpga_util1}
\end{table}

%%%%%%%%%%%%%%%%%%%%%%%%%%%%%%%%%%%%%%%%%%%%%%%%%%
\textbf{Dual 1-Lane Harp Core}\\
Next we instantiate two of the 1-lane Harp cores above to form a multi-core system. Table~\ref{table:perf2} shows the performance of and Table~\ref{table:fpga_util2} shows the logic utilization of this system.

\begin{table}[!htbp]
  \centering
  \begin{tabular}{|l|l|c|c|c|}
    \hline
Core-1 Benchmark	&Core-2 Benchmark	&Instructions 	&Cycles		&IPC	\\
    \hline
Vector Sum   	&Sieve: 			&4523		&9598		&0.4712	\\
Bubble Sort   	&Sieve:     			&2410		&5504		&0.4378	\\
Vector Sum	&Matrix multiplication:		&8994		&19857		&0.4529	\\
Matrix multiplication&Matrix multiplication:	&12164		&19857		&0.6125 \\
    \hline
  \end{tabular}
  \caption{Dual Core 1-Lane Performance}
  \label{table:perf2}
\end{table}

\begin{table}[!htbp]
  \centering
  \begin{tabular}{|l|c|}
    \hline
    \multicolumn{2}{|c|}{Logic Utilization} \\
    \hline
FPGA Utilization:				&4\%  (3,661 / 113,600 ALUTs)\\
Combinational ALUTs:			&13\%       (14,446 / 113,600 ALUTs)\\
Total block memory bits:	&27\%   (1,505,792 / 5,630,976 )\\
    \hline
  \end{tabular}
  \caption{Logic Utilization.}
  \label{table:fpga_util2}
\end{table}

%%%%%%%%%%%%%%%%%%%%%%%%%%%%%%%%%%%%%%%%%%%%%%%%%%
\textbf{Single SIMD 8-Lane Harp Core}\\
We also design and run an 8 Lane SIMD Core on the FPGA board. We used simple applications initially to test the complex coalescing unit. Once that was done we tried to run a compute internsive matrix multiplication code on the board. Comparing the performance numbers we can clearly see the advantage of using SIMD. This comes at the cost of much higher logic utilization due to the coalescing unit and the duplication of ALU blocks and register files for the SIMD core. Table~\ref{table:perf2} shows the performance of and Table~\ref{table:fpga_util3} shows the logic utilization of this system.

%Benchmarks Results:
\begin{table}[!htbp]
  \centering
  \begin{tabular}{|l|c|c|c|c|}
    \hline
Benchmark		&Instructions 	&Cycles		&IPC		&Description\\
    \hline
Matrix Multiplication:					&N/A (N/A) &Multiply 2 matrices of size 16x16 \\
Coalesced Vector Sum:					&N/A (N/A) &Sum 100 numbers doing coalsced access\\
Un-Coalesced Vector Sum:				&N/A (N/A) &Sum 100 numbers doing un-coalsced access\\
    \hline
  \end{tabular}
  \caption{Single Core SIMD 8-Lane Performance}
  \label{table:perf3}
\end{table}

%Logic Consumption:
\begin{table}[!htbp]
  \centering
  \begin{tabular}{|l|c|}
    \hline
    \multicolumn{2}{|c|}{Logic Utilization} \\
    \hline
FPGA Utilization:				&4\%  (3,661 / 113,600 ALUTs)\\
Combinational ALUTs:			&13\%       (14,446 / 113,600 ALUTs)\\
Total block memory bits:	&27\%   (1,505,792 / 5,630,976 )\\
    \hline
  \end{tabular}
  \caption{Logic Utilization.}
  \label{table:fpga_util3}
\end{table}
%%%%%%%%%%%%%%%%%%%%%%%%%%%%%%%%%%%%%%%%%%%%%%%%%%
