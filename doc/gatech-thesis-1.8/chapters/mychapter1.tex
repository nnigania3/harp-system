\setcounter{equation}{0}

\chapter{Introduction}
\label{chap:introduction}

\section{Motivation}
We are building a fast FPGA prototyping tool chain to explore various GPGPU based architectures. The tool chain allows us to also use the same code synthesized to verilog to be used for our simulation flow as well with SST.
Field Programmable Gate Arrays have been used to prototype hardware designs or to do emulation. Since they are hardware implementation of the design they are very fast to run tests and simulations before actually taping out an ASIC. FPGA are used standalone also to implement highly parallel and configurable application specific designs. Having this emulation platform will allow us to run full feature applications which are generally the problem when running software simulations. For example to run an application on the simulator (MACSIM used in this study, \cite{schreier}) we first need to trace the application and then run the trace on the simulator. Since the simulator is slow we trace only a small portion (hot loop) of the application. Whereas on the FPGA we can run the whole app without having to generate traces. For an application which has say a billion instructions running on the simulator like MACSIM which runs at generally 50 kips takes around 5hours where as running on the FPGA (~200 MHz) might take us only a few seconds. The biggest difference or downside when running applications on an FPGA rather than an ASIC is the speed of the FPGA logic, for example the Altera Stratix III FPGA used for this work can run only till about ~500MHz where as an ASIC can run at much higher speed. The modern day FPGA tools make prototyping (implementation and debugging) a painless task as compared to ASIC flow hence this approach was taken.\\
\\
The aim of this wok was to create a tool chain to prototype general purpose graphics processing unit on FPGAs. This hardware flow combined with the higher level simulation flow using the same source code creates this whole tool chain to study future architectures using new technologies quickly. Most of the work in research has focused on either the simulation flow or the hardware flow. Having this flow for research will help to quickly explore hardware designs and show results on the real prototype. Although there are many research works who do publish results for simulation and hardware but this is generally a very long process.

\section{Organization}
The thesis is organized as follows:
\\ 
\noindent\textbf{Chapter 2} gives an overview of the overall system which was prototyped. It also discusses about the various tools used and also gives an overview of CHDL which is the programming language used to write the system design code.
\\
\noindent\textbf{Chapter 3} Describes each of the system components in more detail. We mainly cover some details of the ISA used, the design of the core, the cache and the memory controller. Many obvious micro-architecture details have been skipped to keep the descriptions brief. After discussing about the system components we then discuss about how we went on to integrate all these components and create example designs.
\\
\noindent\textbf{Chapter 4} shows the simulation setup along with some of the results obtained. We also show the resource consumption by the design on the FPGA board.
\\
\noindent\textbf{Chapter 5} finally concludes the work and discusses future directions. 

