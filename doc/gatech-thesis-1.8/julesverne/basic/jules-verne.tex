\documentclass{gatech-thesis}
% The default options:
%   \documentclass[11pt,letterpaper,oneside,%
%       doublespaced,normalmargins,final]{gatech-thesis}
% will generate a document that conforms to the graduate studies office
% guidelines.

 \ifx\pdfoutput\undefined
   \usepackage[dvips,final]{graphicx}  % using latex+dvips
   \usepackage[dvips,usenames]{color}
 \else
   \usepackage[pdftex,final]{graphicx} % using pdflatex
   \usepackage[pdftex,usenames]{color}
 \fi

%define stuff in preamble
 \degree{Doctor of Philosophy}
 \department{School of Electrical and Computer Engineering}
 \title{20,000 Leagues Under the Sea \protect\\ by Jules Verne}
 \author{Frodo Baggins}
 \principaladvisor{Elrond Halfelven}
 \committeechair{Gandalf Ol\'orin}
 \firstreader{Samwise Gamgee}
 \secondreader{Peregrin Took}[School of Industrial and Systems Engineering]
 \thirdreader{Meriadoc Brandybuck}[Department of Combinatorics and Optimization][University of Waterloo]
 \submitdate{August 2002}
 \approveddate{21 August 2002}
 \copyrightyear{2002} %add one if thesis submitted in Dec.

% \thesisproposalfalse       % default
% \titlepagetrue             % default
% \signaturepagetrue         % default
% \copyrightfalse            % default
% \figurespagetrue           % default
% \tablespagetrue            % default
% \contentspagetrue          % default
% \dedicationheadingfalse    % default
% \bibpagetrue               % default
% \strictmarginstrue         % default

\bibfiles{jules-verne-bib}

\begin{document}
\bibliographystyle{gatech-thesis}
\setchaptertocdepth{2}
%
\begin{preliminary}
\begin{dedication}
\begin{center}
  A dedication would go here...
\end{center}

\end{dedication}
\begin{acknowledgements}
Acknowledgement text goes here.  You could also (optionally)
have a forward or a preface, instead of an acknowledgement.

The text used as ``filler'' for Chapters 1, 2, 3, and 
Appendix A and B in this example was taken from the Project
Gutenberg edition of Jules Verne's \textit{20,000 Leagues Under the Sea.}
That text can be freely distributed according to the Gutenberg
license, in any markup language desired (here, \LaTeX) --- so long
as the original text version is also available.  To abide by
these requirements, the \textbf{full} original etext of
\textit{20,000 Leagues} is included in this archive as \texttt{2000010.txt}.

Check \texttt{2000010.txt} for Gutenberg licence and distribution
information.

This class file, \texttt{gatech-thesis.cls}, would not have been
possible without the following people:
\begin{itemize}
\itemsep 0in
\parsep 0in
\item Francois Pitt --- \texttt{ut-thesis.cls} version 1.8, 1999 Dec 10
\item Aichen Low --- original modifications to \texttt{ut-thesis.cls} to 
minimally conform to the Georgia Tech requirements.  Also, her
\texttt{gt-thesissty.sty} provided additional useful 
code that was incorporated into \texttt{gatech-thesis.cls}.
\item Stanford University --- suthesis.sty was adapted by:
\item Ahmed Gheith --- adapted \texttt{suthesis.sty} to create 
\texttt{GTthesis.sty}.  \texttt{GTthesis.sty} was the previous ``official''
Georgia Tech style for dissertations.
\item Joonwon Lee, Eilin Tien Lin, and Wei Lui --- modifications to Ahmed's
original \texttt{GTthesis.sty}
\item Kalyan Perumalla --- additional modifications to \texttt{GTthesis.sty}
(1999 Nov 18)
\item Cody Watson, David Swanson --- even more modifications to 
\texttt{GTthesis.sty}
\end{itemize}
\end{acknowledgements}
%
\contents
%
\begin{summary}
A brief summary of the entire dissertation should go here.
\end{summary}
\end{preliminary}
%
\input{jules-verne-chapter1}
\input{jules-verne-chapter2}
\input{jules-verne-chapter3}
%
\appendix
%
\input{jules-verne-chapter4}
\input{jules-verne-chapter5}
%
\begin{postliminary}
\references
\begin{vita}
Jules Verne was a really good writer from the Nineteenth century.  He wrote a number of
novels which today would be classified as ``science fiction,'' including \textit{From the
Earth to the Moon} and \textit{Around the World in 80 days}, as well as \textit{20,000 
Leagues Under the Sea}.

\end{vita}
\end{postliminary}
\end{document}
